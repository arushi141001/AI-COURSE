\documentclass[10pt,a4paper,twoside]{article}
\usepackage[dutch]{babel}
\usepackage{amssymb}
\usepackage{amsmath}
\usepackage{float,flafter}	
\usepackage{hyperref}
\usepackage{inputenc}
\setlength\paperwidth{20.999cm}\setlength\paperheight{29.699cm}\setlength\voffset{-1in}\setlength\hoffset{-1in}\setlength\topmargin{1cm}\setlength\headheight{12pt}\setlength\headsep{0cm}\setlength\footskip{1.131cm}\setlength\textheight{25cm}\setlength\oddsidemargin{2.499cm}\setlength\textwidth{15.999cm}

\begin{document}
\begin{center}
\hrule

\vspace{.4cm}
{\bf {\Large ASSIGNMENT-4 }}\\
\vspace{.3cm}
{\bf {\huge Can AI be used to understand facts about the Social Sciences?}}
\vspace{.3cm}
\end{center}
{\bf Name:}  Arushi Chaturvedi\\
{\bf Roll no:}  19111013 \\
{\bf Branch: }  Biomedical Engineering \hspace{\fill}  6th August, 2021 \\
\hrule

\vspace{.5cm}
\section{AI Applications in Social Sciences.} 

At a time when social scientists and researchers are rallying for that “human touch” to make AI systems unbiased, we cast a look on how social sciences and humanities can contribute to AI ecosystem and even out the imbalances and biases.
Now, tech giants such as Google, Microsoft, and other companies hire people from social sciences or humanities background.
How Social Sciences Can Make AI More ‘Human’ What makes humans ‘human’ and the goal of any AI, is our unique response to stimuli.
It is predictable and unpredictable at the same time, blasé and responsive all at once.
After much deliberation, it was implied that in the AI field it does not matter how accurate the algorithms are, humans still use their “fast” (unpredictable) in their decisions.

\section{Intuition In Artificial Intelligence}

In a paper titled Explanation in Artificial Intelligence: Insights from the Social Sciences by Tim Miller, the researcher explains that more and more practitioners of AI have been asking for ‘explainable’ AI.
“It is fair to say that most work in explainable AI uses only the researchers’ intuition of what constitutes a ‘good’ explanation.
There are vast and valuable bodies of research in philosophy, psychology, and cognitive science of how people define, generate, select, evaluate, and present explanations, which argues that people employ certain cognitive biases and social expectations towards the explanation process,” says Miller in the paper.
In his paper, Miller argues AI should be built on relevant research from philosophy, cognitive psychology and social psychology.
Realistically, social sciences can be successfully applied in areas such as public policy, human resources, criminology, marketing and advertising, and strategic planning in the public and private sectors.

\section{Points to Ponder}

By considering the current state of social sciences, it is possible to indicate some possible applications of Artificial Intelligence towards problem-solving as follows:
  Employment of intelligent techniques instead of current statistical approaches
 Although Statistics is a strong field, which is widely used along social sciences, it is sometimes more effective to use practical, intelligent approaches of Artificial Intelligence to analyze the objective data.
It is important to improve effectiveness of the analysis, provide more practical and fast solution approaches and even innovative solutions for some problems, which cannot be solved accurately via traditional, statistical methods.
 Running some effective approaches like classification or prediction: Techniques of Artificial Intelligence are generally effective and efficient on achieving classification and prediction operations over objective data.
 Tracking objective objects: The concept of tracking used here is for explaining the effectiveness of Artificial Intelligence in analyzing big amounts of data, which is often associated with social science oriented issues.
At this point, widely known intelligent optimization algorithms of Artificial Intelligence can be applied accordingly.
 Expert System Support: Social sciences often benefit from expert’s knowledge to think and provide solutions.

\end{document}
