\documentclass{article}
\usepackage[utf8]{inputenc}

\title{Gödel's incompleteness theorems}
\author{Arushi Chaturvedi}
\date{ 17 July 2021}

\begin{document}

\maketitle

\section{Summary}
Employing a diagonal argument, Gödel's incompleteness theorems were the first of several closely related theorems on the limitations of formal systems. They were followed by Tarski's undefinability theorem on the formal undefinability of truth
For any such consistent formal system, there will always be statements about natural numbers that are true, but that are unprovable within the system.
The second incompleteness theorem, an extension of the first, shows that the system cannot demonstrate its own consistency.

\subsection{Effective Axiomatization}


Examples of effectively generated theories include Peano arithmetic and Zermelo–Fraenkel set theory . The theory known as true arithmetic consists of all true statements about the standard integers in the language of Peano arithmetic. This theory is consistent, and complete, and contains a sufficient amount of arithmetic.

\subsection{Completeness}

A set of axioms is complete if, for any statement in the axioms' language, that statement or its negation is provable from the axioms . But in a system of mathematics, thinkers such as Hilbert had believed that it is just a matter of time to find such an axiomatization that would allow one to either prove or disprove each and every mathematical formula. A formal system might be syntactically incomplete by design, as logics generally are.

Or it may be incomplete simply because not all the necessary axioms have been discovered or included.  Thus by the first incompleteness theorem, Peano Arithmetic is not complete.

The theorem gives an explicit example of a statement of arithmetic that is neither provable nor disprovable in Peano's arithmetic. In addition, no effectively axiomatized, consistent extension of Peano arithmetic can be complete.

\subsection{Consistency}
Axioms-Consistent  If no statement such that both the statement and its negation are provable from the axioms and inconsistent otherwise.
Peano Arithmetic-ZFC
ZFC-ZFC
Inconsistent theories- Paradoxes

\subsection{System which contain arithmetic}
The theory of algebraically closed fields of a given characteristic is complete, consistent, and has an infinite but recursively enumerable set of axioms.

\subsection{Conflicting goals}
Correct results more than incorrect results.(principle of explosion)\par




\textbf{FIRST INCOMPLETENESS THEOREM} \par


"Any consistent formal system F within which a certain amount of elementary arithmetic can be carried out is incomplete; i.e., there are statements of the language of F which can neither be proved nor disproved in F."\par


To prove the first incompleteness theorem, Gödel demonstrated that the notion of provability within a system could be expressed purely in terms of arithmetical functions that operate on Gödel numbers of sentences of the system


The first incompleteness theorem shows that the Gödel sentence Gf of an appropriate formal theory F is unprovable in F.



\textbf{SECOND INCOMPLETENESS THEOREM}

 "Assume F is a consistent formalized system which contains elementary arithmetic. 



 The proof of the second incompleteness theorem is obtained by formalizing the proof of the first incompleteness theorem within the system F itself.

Exrpresses Consistency

Hilbert-Bernays Conditions



This theorem is stronger than the first incompleteness theorem because the statement constructed in the first incompleteness theorem does not directly express the consistency of the system. The proof of the second incompleteness theorem is obtained by formalizing the proof of the first incompleteness theorem within the system F itself.
His incompleteness theorems meant there can be no mathematical theory of everything, no unification of what’s provable and what’s true. What mathematicians can prove depends on their starting assumptions, not on any fundamental ground truth from which all answers spring.





The main difficulty in proving the second incompleteness theorem is to show that various facts about provability used in the proof of the first incompleteness theorem can be formalized within the system using a formal predicate for provability. Once this is done, the second incompleteness theorem follows by formalizing the entire proof of the first incompleteness theorem within the system itself.
\begin{itemize}

\item Logicism and Hilbert's second problem
\item  Minds and Machines
\item Paraconsistent Logic
\item Appeals to the incompleteness theorems in other fields(beyond mathematics and logic)

\end{itemize}


 




















\end{document}