\documentclass[10pt,a4paper,twoside]{article}
\usepackage[dutch]{babel}
\usepackage{amssymb}
\usepackage{amsmath}
\usepackage{float,flafter}	
\usepackage{hyperref}
\usepackage{inputenc}
\setlength\paperwidth{20.999cm}\setlength\paperheight{29.699cm}\setlength\voffset{-1in}\setlength\hoffset{-1in}\setlength\topmargin{1cm}\setlength\headheight{12pt}\setlength\headsep{0cm}\setlength\footskip{1.131cm}\setlength\textheight{25cm}\setlength\oddsidemargin{2.499cm}\setlength\textwidth{15.999cm}

\begin{document}
\begin{center}
\hrule

\vspace{.4cm}
{\bf {\Large ASSIGNMENT-3 }}\\
\vspace{.3cm}
{\bf {\huge Moravec's Paradox}}
\vspace{.3cm}
\end{center}
{\bf Name:}  Arushi Chaturvedi\\
{\bf Roll no:}  19111013 \\
{\bf Branch: }  Biomedical Engineering \hspace{\fill}  20 July, 2021 \\
\hrule

\vspace{.5cm}
\section{Summary.} 

Moravec's paradox is the observation by artificial intelligence and robotics researchers that, contrary to traditional assumptions, reasoning requires very little computation, but sensorimotor skills require enormous computational resources.
In general, we're least aware of what our minds do best", he wrote, and added "we're more aware of simple processes that don't work well than of complex ones that work flawlessly"

\section{The Biological Basis of human skills}
One possible explanation of the paradox, offered by Moravec, is based on evolution. natural selection has tended to preserve design improvements and optimizations.We should expect the difficulty of reverse-engineering any human skill to be roughly proportional to the amount of time that skill has been evolving in animals.
Eg: recognizing a face, motor skills, social skills etc.


\section{Historical Influence on Artificial Intelligence}
Logic and algebra: sign of intelligence(wrong). Rodney Brooks explains,best characterized as the things that highly educated male scientists found challenging.(chess,symbolic integration etc).He decided to build intelligent machines that had "No cognition. This new direction, which he called "Nouvelle AI" was highly influential on robotics research and AI.

\section{Result}
35 years of AI research : Hard problems are easy and easy problems are hard.

\end{document}
